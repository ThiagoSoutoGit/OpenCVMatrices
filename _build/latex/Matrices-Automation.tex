%% Generated by Sphinx.
\def\sphinxdocclass{report}
\documentclass[letterpaper,10pt,english]{sphinxmanual}
\ifdefined\pdfpxdimen
   \let\sphinxpxdimen\pdfpxdimen\else\newdimen\sphinxpxdimen
\fi \sphinxpxdimen=.75bp\relax

\PassOptionsToPackage{warn}{textcomp}
\usepackage[utf8]{inputenc}
\ifdefined\DeclareUnicodeCharacter
 \ifdefined\DeclareUnicodeCharacterAsOptional
  \DeclareUnicodeCharacter{"00A0}{\nobreakspace}
  \DeclareUnicodeCharacter{"2500}{\sphinxunichar{2500}}
  \DeclareUnicodeCharacter{"2502}{\sphinxunichar{2502}}
  \DeclareUnicodeCharacter{"2514}{\sphinxunichar{2514}}
  \DeclareUnicodeCharacter{"251C}{\sphinxunichar{251C}}
  \DeclareUnicodeCharacter{"2572}{\textbackslash}
 \else
  \DeclareUnicodeCharacter{00A0}{\nobreakspace}
  \DeclareUnicodeCharacter{2500}{\sphinxunichar{2500}}
  \DeclareUnicodeCharacter{2502}{\sphinxunichar{2502}}
  \DeclareUnicodeCharacter{2514}{\sphinxunichar{2514}}
  \DeclareUnicodeCharacter{251C}{\sphinxunichar{251C}}
  \DeclareUnicodeCharacter{2572}{\textbackslash}
 \fi
\fi
\usepackage{cmap}
\usepackage[T1]{fontenc}
\usepackage{amsmath,amssymb,amstext}
\usepackage{babel}
\usepackage{times}
\usepackage[Bjarne]{fncychap}
\usepackage{sphinx}

\usepackage{geometry}

% Include hyperref last.
\usepackage{hyperref}
% Fix anchor placement for figures with captions.
\usepackage{hypcap}% it must be loaded after hyperref.
% Set up styles of URL: it should be placed after hyperref.
\urlstyle{same}
\addto\captionsenglish{\renewcommand{\contentsname}{Contents:}}

\addto\captionsenglish{\renewcommand{\figurename}{Fig.}}
\addto\captionsenglish{\renewcommand{\tablename}{Table}}
\addto\captionsenglish{\renewcommand{\literalblockname}{Listing}}

\addto\captionsenglish{\renewcommand{\literalblockcontinuedname}{continued from previous page}}
\addto\captionsenglish{\renewcommand{\literalblockcontinuesname}{continues on next page}}

\addto\extrasenglish{\def\pageautorefname{page}}

\setcounter{tocdepth}{1}



\title{Matrices - Automation Documentation}
\date{Apr 23, 2020}
\release{1.0}
\author{Thiago Souto}
\newcommand{\sphinxlogo}{\vbox{}}
\renewcommand{\releasename}{Release}
\makeindex

\begin{document}

\maketitle
\sphinxtableofcontents
\phantomsection\label{\detokenize{index::doc}}



\chapter{MatrixManipulation module}
\label{\detokenize{rst/MatrixManipulation::doc}}\label{\detokenize{rst/MatrixManipulation:welcome-to-matrices-automation-s-documentation}}\label{\detokenize{rst/MatrixManipulation:matrixmanipulation-module}}\label{\detokenize{rst/MatrixManipulation:module-MatrixManipulation}}\index{MatrixManipulation (module)}\index{Matrix (class in MatrixManipulation)}

\begin{fulllineitems}
\phantomsection\label{\detokenize{rst/MatrixManipulation:MatrixManipulation.Matrix}}\pysiglinewithargsret{\sphinxbfcode{\sphinxupquote{class }}\sphinxcode{\sphinxupquote{MatrixManipulation.}}\sphinxbfcode{\sphinxupquote{Matrix}}}{\emph{**kwargs}}{}
Bases: \sphinxhref{https://docs.python.org/3/library/functions.html\#object}{\sphinxcode{\sphinxupquote{object}}}

Definition: This class generates Rotation and Translation matrices,
that can be used to multiply any matrix and obtain the translation or rotation.

It uses \sphinxtitleref{numpy} to generate the matrices:
\begin{quote}

np.float32: creates the array with 16 float32 elements

np.reshape: np.reshape rearrange the array into a 4X4 matrix
\end{quote}

Returns: It returns Rotation and translation matrices.

Obs: {\color{red}\bfseries{}**}kwargs (keyword arguments) are used to facilitate the identification of the parameters, so initiate the
object
like: Matrix(x\_angle=’45’, x\_dist=’100’, z\_angle=’60’, z\_dist=’100’), if an argument is not provided,
the default 0 will be put to the argument.
\index{rot\_x() (MatrixManipulation.Matrix method)}

\begin{fulllineitems}
\phantomsection\label{\detokenize{rst/MatrixManipulation:MatrixManipulation.Matrix.rot_x}}\pysiglinewithargsret{\sphinxbfcode{\sphinxupquote{rot\_x}}}{\emph{gamma=0}, \emph{degrees=True}}{}
Definition: Receives an alpha angle and returns the rotation matrix for the given angle at the \sphinxstyleemphasis{X} axis.
If the angle is given in radian degrees should be False.
\begin{quote}\begin{description}
\item[{Parameters}] \leavevmode\begin{itemize}
\item {} 
\sphinxstyleliteralstrong{\sphinxupquote{gamma}} (\sphinxhref{https://docs.python.org/3/library/functions.html\#float}{\sphinxstyleliteralemphasis{\sphinxupquote{float}}}) \textendash{} Rotation Angle around the X axis

\item {} 
\sphinxstyleliteralstrong{\sphinxupquote{degrees}} (\sphinxhref{https://docs.python.org/3/library/functions.html\#bool}{\sphinxstyleliteralemphasis{\sphinxupquote{bool}}}) \textendash{} Indicates if the provided angle is in degrees, if yes It will be converted to radians

\end{itemize}

\end{description}\end{quote}

Returns: The Rotational Matrix at the X axis by an \sphinxstyleemphasis{gamma} angle

\end{fulllineitems}

\index{rot\_y() (MatrixManipulation.Matrix method)}

\begin{fulllineitems}
\phantomsection\label{\detokenize{rst/MatrixManipulation:MatrixManipulation.Matrix.rot_y}}\pysiglinewithargsret{\sphinxbfcode{\sphinxupquote{rot\_y}}}{\emph{beta=0}, \emph{degrees=True}}{}
Definition: Receives an theta angle and returns the rotation matrix for the given angle at the \sphinxstyleemphasis{Z} axis.
If the angle is given in radian degrees should be False.
\begin{quote}\begin{description}
\item[{Parameters}] \leavevmode\begin{itemize}
\item {} 
\sphinxstyleliteralstrong{\sphinxupquote{beta}} (\sphinxhref{https://docs.python.org/3/library/functions.html\#float}{\sphinxstyleliteralemphasis{\sphinxupquote{float}}}) \textendash{} Rotation Angle around the Z axis

\item {} 
\sphinxstyleliteralstrong{\sphinxupquote{degrees}} (\sphinxhref{https://docs.python.org/3/library/functions.html\#bool}{\sphinxstyleliteralemphasis{\sphinxupquote{bool}}}) \textendash{} Indicates if the provided angle is in degrees, if yes It will be converted to radians

\end{itemize}

\end{description}\end{quote}

Returns: The Rotational Matrix at the Z axis by an \sphinxstyleemphasis{beta} angle

\end{fulllineitems}

\index{rot\_z() (MatrixManipulation.Matrix method)}

\begin{fulllineitems}
\phantomsection\label{\detokenize{rst/MatrixManipulation:MatrixManipulation.Matrix.rot_z}}\pysiglinewithargsret{\sphinxbfcode{\sphinxupquote{rot\_z}}}{\emph{alpha=0}, \emph{degrees=True}}{}
Definition: Receives an theta angle and returns the rotation matrix for the given angle at the \sphinxstyleemphasis{Z} axis.
If the angle is given in radian degrees should be False.
\begin{quote}\begin{description}
\item[{Parameters}] \leavevmode\begin{itemize}
\item {} 
\sphinxstyleliteralstrong{\sphinxupquote{alpha}} (\sphinxhref{https://docs.python.org/3/library/functions.html\#float}{\sphinxstyleliteralemphasis{\sphinxupquote{float}}}) \textendash{} Rotation Angle around the Z axis

\item {} 
\sphinxstyleliteralstrong{\sphinxupquote{degrees}} (\sphinxhref{https://docs.python.org/3/library/functions.html\#bool}{\sphinxstyleliteralemphasis{\sphinxupquote{bool}}}) \textendash{} Indicates if the provided angle is in degrees, if yes It will be converted to radians

\end{itemize}

\end{description}\end{quote}

Returns: The Rotational Matrix at the Z axis by an \sphinxstyleemphasis{alpha} angle

\end{fulllineitems}

\index{trans\_x() (MatrixManipulation.Matrix method)}

\begin{fulllineitems}
\phantomsection\label{\detokenize{rst/MatrixManipulation:MatrixManipulation.Matrix.trans_x}}\pysiglinewithargsret{\sphinxbfcode{\sphinxupquote{trans\_x}}}{\emph{a=0}}{}
Definition: Translates the matrix a given amount \sphinxtitleref{a} on the \sphinxstyleemphasis{X} axis by Defining a 4x4 identity
matrix with \sphinxtitleref{a} as the (1,4) element.
\begin{quote}\begin{description}
\item[{Parameters}] \leavevmode
\sphinxstyleliteralstrong{\sphinxupquote{a}} (\sphinxhref{https://docs.python.org/3/library/functions.html\#float}{\sphinxstyleliteralemphasis{\sphinxupquote{float}}}) \textendash{} Distance translated on the X-axis

\end{description}\end{quote}

Returns: The Translation Matrix on the \sphinxstyleemphasis{X} axis by a distance \sphinxstyleemphasis{a}

\end{fulllineitems}

\index{trans\_y() (MatrixManipulation.Matrix method)}

\begin{fulllineitems}
\phantomsection\label{\detokenize{rst/MatrixManipulation:MatrixManipulation.Matrix.trans_y}}\pysiglinewithargsret{\sphinxbfcode{\sphinxupquote{trans\_y}}}{\emph{b=0}}{}
Definition: Translate the matrix a given amount \sphinxtitleref{d} on the \sphinxstyleemphasis{Z} axis. by Defining a matrix T 4x4 identity
matrix with \sphinxstyleemphasis{b} (3,4) element position.
\begin{quote}\begin{description}
\item[{Parameters}] \leavevmode
\sphinxstyleliteralstrong{\sphinxupquote{b}} (\sphinxhref{https://docs.python.org/3/library/functions.html\#float}{\sphinxstyleliteralemphasis{\sphinxupquote{float}}}) \textendash{} Distance translated on the Z-axis

\end{description}\end{quote}

Returns: The Translation Matrix on the \sphinxstyleemphasis{Z} axis by a distance \sphinxstyleemphasis{b}

\end{fulllineitems}

\index{trans\_z() (MatrixManipulation.Matrix method)}

\begin{fulllineitems}
\phantomsection\label{\detokenize{rst/MatrixManipulation:MatrixManipulation.Matrix.trans_z}}\pysiglinewithargsret{\sphinxbfcode{\sphinxupquote{trans\_z}}}{\emph{c=0}}{}
Definition: Translate the matrix a given amount \sphinxtitleref{d} on the \sphinxstyleemphasis{Z} axis. by Defining a matrix T 4x4 identity
matrix with \sphinxstyleemphasis{c} (3,4) element position.
\begin{quote}\begin{description}
\item[{Parameters}] \leavevmode
\sphinxstyleliteralstrong{\sphinxupquote{c}} (\sphinxhref{https://docs.python.org/3/library/functions.html\#float}{\sphinxstyleliteralemphasis{\sphinxupquote{float}}}) \textendash{} Distance translated on the Z-axis

\end{description}\end{quote}

Returns: The Translation Matrix on the \sphinxstyleemphasis{Z} axis by a distance \sphinxstyleemphasis{c}

\end{fulllineitems}


\end{fulllineitems}

\index{main() (in module MatrixManipulation)}

\begin{fulllineitems}
\phantomsection\label{\detokenize{rst/MatrixManipulation:MatrixManipulation.main}}\pysiglinewithargsret{\sphinxcode{\sphinxupquote{MatrixManipulation.}}\sphinxbfcode{\sphinxupquote{main}}}{}{}
Example 3

\end{fulllineitems}



\chapter{Indices and tables}
\label{\detokenize{index:indices-and-tables}}\begin{itemize}
\item {} 
\DUrole{xref,std,std-ref}{genindex}

\item {} 
\DUrole{xref,std,std-ref}{modindex}

\item {} 
\DUrole{xref,std,std-ref}{search}

\end{itemize}


\section{Running the documentation with Sphinx}
\label{\detokenize{index:running-the-documentation-with-sphinx}}
To run the documentation for this project run the following commands, at the project folder:
\begin{quote}

Install Spinxs:

\sphinxstylestrong{python -m pip install sphinx}

Install the “Read the Docs” theme:

\sphinxstylestrong{pip install sphinx-rtd-theme}

\sphinxstylestrong{make clean}

\sphinxstylestrong{make html}
\end{quote}


\renewcommand{\indexname}{Python Module Index}
\begin{sphinxtheindex}
\def\bigletter#1{{\Large\sffamily#1}\nopagebreak\vspace{1mm}}
\bigletter{m}
\item {\sphinxstyleindexentry{MatrixManipulation}}\sphinxstyleindexpageref{rst/MatrixManipulation:\detokenize{module-MatrixManipulation}}
\end{sphinxtheindex}

\renewcommand{\indexname}{Index}
\printindex
\end{document}