%% Generated by Sphinx.
\def\sphinxdocclass{report}
\documentclass[letterpaper,10pt,english]{sphinxmanual}
\ifdefined\pdfpxdimen
   \let\sphinxpxdimen\pdfpxdimen\else\newdimen\sphinxpxdimen
\fi \sphinxpxdimen=.75bp\relax

\PassOptionsToPackage{warn}{textcomp}
\usepackage[utf8]{inputenc}
\ifdefined\DeclareUnicodeCharacter
 \ifdefined\DeclareUnicodeCharacterAsOptional
  \DeclareUnicodeCharacter{"00A0}{\nobreakspace}
  \DeclareUnicodeCharacter{"2500}{\sphinxunichar{2500}}
  \DeclareUnicodeCharacter{"2502}{\sphinxunichar{2502}}
  \DeclareUnicodeCharacter{"2514}{\sphinxunichar{2514}}
  \DeclareUnicodeCharacter{"251C}{\sphinxunichar{251C}}
  \DeclareUnicodeCharacter{"2572}{\textbackslash}
 \else
  \DeclareUnicodeCharacter{00A0}{\nobreakspace}
  \DeclareUnicodeCharacter{2500}{\sphinxunichar{2500}}
  \DeclareUnicodeCharacter{2502}{\sphinxunichar{2502}}
  \DeclareUnicodeCharacter{2514}{\sphinxunichar{2514}}
  \DeclareUnicodeCharacter{251C}{\sphinxunichar{251C}}
  \DeclareUnicodeCharacter{2572}{\textbackslash}
 \fi
\fi
\usepackage{cmap}
\usepackage[T1]{fontenc}
\usepackage{amsmath,amssymb,amstext}
\usepackage{babel}
\usepackage{times}
\usepackage[Bjarne]{fncychap}
\usepackage{sphinx}

\usepackage{geometry}

% Include hyperref last.
\usepackage{hyperref}
% Fix anchor placement for figures with captions.
\usepackage{hypcap}% it must be loaded after hyperref.
% Set up styles of URL: it should be placed after hyperref.
\urlstyle{same}
\addto\captionsenglish{\renewcommand{\contentsname}{Contents:}}

\addto\captionsenglish{\renewcommand{\figurename}{Fig.}}
\addto\captionsenglish{\renewcommand{\tablename}{Table}}
\addto\captionsenglish{\renewcommand{\literalblockname}{Listing}}

\addto\captionsenglish{\renewcommand{\literalblockcontinuedname}{continued from previous page}}
\addto\captionsenglish{\renewcommand{\literalblockcontinuesname}{continues on next page}}

\addto\extrasenglish{\def\pageautorefname{page}}

\setcounter{tocdepth}{1}



\title{Matrices - Automation Documentation}
\date{Apr 23, 2020}
\release{1.0}
\author{Thiago Souto}
\newcommand{\sphinxlogo}{\vbox{}}
\renewcommand{\releasename}{Release}
\makeindex

\begin{document}

\maketitle
\sphinxtableofcontents
\phantomsection\label{\detokenize{index::doc}}



\chapter{Example\_6 module}
\label{\detokenize{rst/Example_6:example-6-module}}\label{\detokenize{rst/Example_6::doc}}\label{\detokenize{rst/Example_6:welcome-to-matrices-automation-s-documentation}}\label{\detokenize{rst/Example_6:module-Example_6}}\index{Example\_6 (module)}\index{T() (in module Example\_6)}

\begin{fulllineitems}
\phantomsection\label{\detokenize{rst/Example_6:Example_6.T}}\pysiglinewithargsret{\sphinxcode{\sphinxupquote{Example\_6.}}\sphinxbfcode{\sphinxupquote{T}}}{\emph{alpha=0}, \emph{a=0}, \emph{d=0}, \emph{theta=0}, \emph{degrees=True}}{}
Definition: Receives four arguments, \sphinxstyleemphasis{alpha} and \sphinxstyleemphasis{a}, being angle for rotation in the X axis and translation on
the X axis. Also \sphinxstyleemphasis{d} and \sphinxstyleemphasis{theta}, being translation on the Z axis and Rotation on the Z axis. And returns the
Multiplication of (Rotation matrix in X and the Translation in X) multiplied by (Rotation matrix in Z and the
Translation in Z) . It utilizes the np.matmul for matrix multiplication.
\begin{quote}\begin{description}
\item[{Parameters}] \leavevmode\begin{itemize}
\item {} 
\sphinxstyleliteralstrong{\sphinxupquote{alpha}} (\sphinxhref{https://docs.python.org/3/library/functions.html\#float}{\sphinxstyleliteralemphasis{\sphinxupquote{float}}}) \textendash{} Rotation Angle around the X axis

\item {} 
\sphinxstyleliteralstrong{\sphinxupquote{a}} (\sphinxhref{https://docs.python.org/3/library/functions.html\#float}{\sphinxstyleliteralemphasis{\sphinxupquote{float}}}) \textendash{} Distance translated on the X-axis

\item {} 
\sphinxstyleliteralstrong{\sphinxupquote{d}} (\sphinxhref{https://docs.python.org/3/library/functions.html\#float}{\sphinxstyleliteralemphasis{\sphinxupquote{float}}}) \textendash{} Distance translated on the Z-axis

\item {} 
\sphinxstyleliteralstrong{\sphinxupquote{theta}} (\sphinxhref{https://docs.python.org/3/library/functions.html\#float}{\sphinxstyleliteralemphasis{\sphinxupquote{float}}}) \textendash{} Rotation Angle around the Z axis

\item {} 
\sphinxstyleliteralstrong{\sphinxupquote{degrees}} (\sphinxhref{https://docs.python.org/3/library/functions.html\#bool}{\sphinxstyleliteralemphasis{\sphinxupquote{bool}}}) \textendash{} Indicates if the provided angle is in degrees, if yes It will be converted to radians

\end{itemize}

\end{description}\end{quote}

Returns: A matrix with the Rotations and translations set.

\end{fulllineitems}

\index{T\_rot\_x() (in module Example\_6)}

\begin{fulllineitems}
\phantomsection\label{\detokenize{rst/Example_6:Example_6.T_rot_x}}\pysiglinewithargsret{\sphinxcode{\sphinxupquote{Example\_6.}}\sphinxbfcode{\sphinxupquote{T\_rot\_x}}}{\emph{alpha=0}, \emph{degrees=True}}{}
Definition: Receives an alpha angle and returns the rotation matrix for the given angle at the \sphinxstyleemphasis{X} axis.
If the angle is given in radian degrees should be False.
\begin{quote}\begin{description}
\item[{Parameters}] \leavevmode\begin{itemize}
\item {} 
\sphinxstyleliteralstrong{\sphinxupquote{alpha}} (\sphinxhref{https://docs.python.org/3/library/functions.html\#float}{\sphinxstyleliteralemphasis{\sphinxupquote{float}}}) \textendash{} Rotation Angle around the X axis

\item {} 
\sphinxstyleliteralstrong{\sphinxupquote{degrees}} (\sphinxhref{https://docs.python.org/3/library/functions.html\#bool}{\sphinxstyleliteralemphasis{\sphinxupquote{bool}}}) \textendash{} Indicates if the provided angle is in degrees, if yes It will be converted to radians

\end{itemize}

\end{description}\end{quote}

Returns: The Rotational Matrix at the X axis by an \sphinxstyleemphasis{alpha} angle

\end{fulllineitems}

\index{T\_rot\_z() (in module Example\_6)}

\begin{fulllineitems}
\phantomsection\label{\detokenize{rst/Example_6:Example_6.T_rot_z}}\pysiglinewithargsret{\sphinxcode{\sphinxupquote{Example\_6.}}\sphinxbfcode{\sphinxupquote{T\_rot\_z}}}{\emph{theta=0}, \emph{degrees=True}}{}
Definition: Receives an theta angle and returns the rotation matrix for the given angle at the \sphinxstyleemphasis{Z} axis.
If the angle is given in radian degrees should be False.
\begin{quote}\begin{description}
\item[{Parameters}] \leavevmode\begin{itemize}
\item {} 
\sphinxstyleliteralstrong{\sphinxupquote{theta}} (\sphinxhref{https://docs.python.org/3/library/functions.html\#float}{\sphinxstyleliteralemphasis{\sphinxupquote{float}}}) \textendash{} Rotation Angle around the Z axis

\item {} 
\sphinxstyleliteralstrong{\sphinxupquote{degrees}} (\sphinxhref{https://docs.python.org/3/library/functions.html\#bool}{\sphinxstyleliteralemphasis{\sphinxupquote{bool}}}) \textendash{} Indicates if the provided angle is in degrees, if yes It will be converted to radians

\end{itemize}

\end{description}\end{quote}

Returns: The Rotational Matrix at the Z axis by an \sphinxstyleemphasis{theta} angle

\end{fulllineitems}

\index{T\_trans\_x() (in module Example\_6)}

\begin{fulllineitems}
\phantomsection\label{\detokenize{rst/Example_6:Example_6.T_trans_x}}\pysiglinewithargsret{\sphinxcode{\sphinxupquote{Example\_6.}}\sphinxbfcode{\sphinxupquote{T\_trans\_x}}}{\emph{a=0}}{}
Definition: Translate the matrix a given amount \sphinxtitleref{a} on the \sphinxstyleemphasis{X} axis by Defining a matrix T 4x4 identity
matrix with \sphinxstyleemphasis{a} (1,4) element position.

np.float32: creates the array with 16 float32 elements

np.reshape: np.reshape rearrange the array into a matrix with 4 lines and 4 columns
\begin{quote}\begin{description}
\item[{Parameters}] \leavevmode
\sphinxstyleliteralstrong{\sphinxupquote{a}} (\sphinxhref{https://docs.python.org/3/library/functions.html\#float}{\sphinxstyleliteralemphasis{\sphinxupquote{float}}}) \textendash{} Distance translated on the X-axis

\end{description}\end{quote}

Returns: The Translation Matrix on the \sphinxstyleemphasis{X} axis by a distance \sphinxstyleemphasis{a}

\end{fulllineitems}

\index{T\_trans\_z() (in module Example\_6)}

\begin{fulllineitems}
\phantomsection\label{\detokenize{rst/Example_6:Example_6.T_trans_z}}\pysiglinewithargsret{\sphinxcode{\sphinxupquote{Example\_6.}}\sphinxbfcode{\sphinxupquote{T\_trans\_z}}}{\emph{d=0}}{}
Definition: Translate the matrix a given amount \sphinxtitleref{d} on the \sphinxstyleemphasis{Z} axis. by Defining a matrix T 4x4 identity
matrix with \sphinxstyleemphasis{d} (3,4) element position.
\begin{quote}\begin{description}
\item[{Parameters}] \leavevmode
\sphinxstyleliteralstrong{\sphinxupquote{d}} (\sphinxhref{https://docs.python.org/3/library/functions.html\#float}{\sphinxstyleliteralemphasis{\sphinxupquote{float}}}) \textendash{} Distance translated on the Z-axis

\end{description}\end{quote}

Returns: The Translation Matrix on the \sphinxstyleemphasis{Z} axis by a distance \sphinxstyleemphasis{d}

\end{fulllineitems}

\index{main() (in module Example\_6)}

\begin{fulllineitems}
\phantomsection\label{\detokenize{rst/Example_6:Example_6.main}}\pysiglinewithargsret{\sphinxcode{\sphinxupquote{Example\_6.}}\sphinxbfcode{\sphinxupquote{main}}}{}{}~\begin{description}
\item[{Definition: Complete a series of operations using the functions defined including:}] \leavevmode
Defines a matrix with no rotation and no translation (Identity)
Translation of a given distance on the X axis
Second Translation of a given distance on the X axis
Identity matrix multiplied by the first X translation multiplied by the second translation
Rotation Matrix in X by a given angle
Rotation Matrix in Z by a given angle
Printe them all

\end{description}

\end{fulllineitems}



\chapter{Example\_6\_Symbolic module}
\label{\detokenize{rst/Example_6_Symbolic:example-6-symbolic-module}}\label{\detokenize{rst/Example_6_Symbolic::doc}}\label{\detokenize{rst/Example_6_Symbolic:module-Example_6_Symbolic}}\index{Example\_6\_Symbolic (module)}\index{main() (in module Example\_6\_Symbolic)}

\begin{fulllineitems}
\phantomsection\label{\detokenize{rst/Example_6_Symbolic:Example_6_Symbolic.main}}\pysiglinewithargsret{\sphinxcode{\sphinxupquote{Example\_6\_Symbolic.}}\sphinxbfcode{\sphinxupquote{main}}}{}{}
This code will prints several Matrices like like Dot products, Rotational Matrix in a symbolic form.
Refer to :ref: Example\_6 for details on the matrices operations.

\end{fulllineitems}



\chapter{Indices and tables}
\label{\detokenize{index:indices-and-tables}}\begin{itemize}
\item {} 
\DUrole{xref,std,std-ref}{genindex}

\item {} 
\DUrole{xref,std,std-ref}{modindex}

\item {} 
\DUrole{xref,std,std-ref}{search}

\end{itemize}


\section{Running the documentation with Sphinx}
\label{\detokenize{index:running-the-documentation-with-sphinx}}
To run the documentation for this project run the following commands, at the project folder:
\begin{quote}

Install Spinxs:

\sphinxstylestrong{python -m pip install sphinx}

Install the “Read the Docs” theme:

\sphinxstylestrong{pip install sphinx-rtd-theme}

\sphinxstylestrong{make clean}

\sphinxstylestrong{make html}
\end{quote}


\renewcommand{\indexname}{Python Module Index}
\begin{sphinxtheindex}
\def\bigletter#1{{\Large\sffamily#1}\nopagebreak\vspace{1mm}}
\bigletter{e}
\item {\sphinxstyleindexentry{Example\_6}}\sphinxstyleindexpageref{rst/Example_6:\detokenize{module-Example_6}}
\item {\sphinxstyleindexentry{Example\_6\_Symbolic}}\sphinxstyleindexpageref{rst/Example_6_Symbolic:\detokenize{module-Example_6_Symbolic}}
\end{sphinxtheindex}

\renewcommand{\indexname}{Index}
\printindex
\end{document}