%% Generated by Sphinx.
\def\sphinxdocclass{report}
\documentclass[letterpaper,10pt,english,openany,oneside]{sphinxmanual}
\ifdefined\pdfpxdimen
   \let\sphinxpxdimen\pdfpxdimen\else\newdimen\sphinxpxdimen
\fi \sphinxpxdimen=.75bp\relax

\PassOptionsToPackage{warn}{textcomp}
\usepackage[utf8]{inputenc}
\ifdefined\DeclareUnicodeCharacter
 \ifdefined\DeclareUnicodeCharacterAsOptional
  \DeclareUnicodeCharacter{"00A0}{\nobreakspace}
  \DeclareUnicodeCharacter{"2500}{\sphinxunichar{2500}}
  \DeclareUnicodeCharacter{"2502}{\sphinxunichar{2502}}
  \DeclareUnicodeCharacter{"2514}{\sphinxunichar{2514}}
  \DeclareUnicodeCharacter{"251C}{\sphinxunichar{251C}}
  \DeclareUnicodeCharacter{"2572}{\textbackslash}
 \else
  \DeclareUnicodeCharacter{00A0}{\nobreakspace}
  \DeclareUnicodeCharacter{2500}{\sphinxunichar{2500}}
  \DeclareUnicodeCharacter{2502}{\sphinxunichar{2502}}
  \DeclareUnicodeCharacter{2514}{\sphinxunichar{2514}}
  \DeclareUnicodeCharacter{251C}{\sphinxunichar{251C}}
  \DeclareUnicodeCharacter{2572}{\textbackslash}
 \fi
\fi
\usepackage{cmap}
\usepackage[T1]{fontenc}
\usepackage{amsmath,amssymb,amstext}
\usepackage{babel}
\usepackage{times}
\usepackage[Bjarne]{fncychap}
\usepackage{sphinx}

\usepackage{geometry}

% Include hyperref last.
\usepackage{hyperref}
% Fix anchor placement for figures with captions.
\usepackage{hypcap}% it must be loaded after hyperref.
% Set up styles of URL: it should be placed after hyperref.
\urlstyle{same}
\addto\captionsenglish{\renewcommand{\contentsname}{Contents:}}

\addto\captionsenglish{\renewcommand{\figurename}{Fig.}}
\addto\captionsenglish{\renewcommand{\tablename}{Table}}
\addto\captionsenglish{\renewcommand{\literalblockname}{Listing}}

\addto\captionsenglish{\renewcommand{\literalblockcontinuedname}{continued from previous page}}
\addto\captionsenglish{\renewcommand{\literalblockcontinuesname}{continues on next page}}

\addto\extrasenglish{\def\pageautorefname{page}}

\setcounter{tocdepth}{1}



\title{Matrices Manipulation Documentation}
\date{Apr 27, 2020}
\release{1.0}
\author{Thiago Souto}
\newcommand{\sphinxlogo}{\vbox{}}
\renewcommand{\releasename}{Release}
\makeindex

\begin{document}

\maketitle
\sphinxtableofcontents
\phantomsection\label{\detokenize{index::doc}}



\chapter{MatrixManipulation module}
\label{\detokenize{MatrixManipulation:matrixmanipulation-module}}\label{\detokenize{MatrixManipulation:welcome-to-matrices-manipulation-s-documentation}}\label{\detokenize{MatrixManipulation:module-MatrixManipulation}}\label{\detokenize{MatrixManipulation::doc}}\index{MatrixManipulation (module)}\index{Matrix (class in MatrixManipulation)}

\begin{fulllineitems}
\phantomsection\label{\detokenize{MatrixManipulation:MatrixManipulation.Matrix}}\pysiglinewithargsret{\sphinxbfcode{\sphinxupquote{class }}\sphinxcode{\sphinxupquote{MatrixManipulation.}}\sphinxbfcode{\sphinxupquote{Matrix}}}{\emph{**kwargs}}{}
Bases: \sphinxhref{https://docs.python.org/3/library/functions.html\#object}{\sphinxcode{\sphinxupquote{object}}}

Definition: This class generates Homogeneous transform matrices,
that can be used to multiply any matrix and obtain the translation or rotation.

It uses \sphinxtitleref{numpy} to generate the matrices:
\begin{quote}

np.float32: creates the array with 16 float32 elements

np.reshape: np.reshape rearrange the array into a 4X4 matrix
\end{quote}

Returns: It returns Rotation and translation matrices.

Obs: {\color{red}\bfseries{}**}kwargs (keyword arguments) are used to facilitate the identification of the parameters, so initiate the
object
like: Matrix(x\_angle=’45’, x\_dist=’100’, z\_angle=’60’, z\_dist=’100’), if an argument is not provided,
the default 0 will be put to the argument.
\index{rot\_x() (MatrixManipulation.Matrix method)}

\begin{fulllineitems}
\phantomsection\label{\detokenize{MatrixManipulation:MatrixManipulation.Matrix.rot_x}}\pysiglinewithargsret{\sphinxbfcode{\sphinxupquote{rot\_x}}}{\emph{gamma=0}, \emph{degrees=True}}{}
Definition: Receives an alpha angle and returns the rotation matrix for the given angle at the \sphinxstyleemphasis{X} axis.
If the angle is given in radian degrees should be False.
\begin{quote}\begin{description}
\item[{Parameters}] \leavevmode\begin{itemize}
\item {} 
\sphinxstyleliteralstrong{\sphinxupquote{gamma}} (\sphinxhref{https://docs.python.org/3/library/functions.html\#float}{\sphinxstyleliteralemphasis{\sphinxupquote{float}}}) \textendash{} Rotation Angle around the X axis

\item {} 
\sphinxstyleliteralstrong{\sphinxupquote{degrees}} (\sphinxhref{https://docs.python.org/3/library/functions.html\#bool}{\sphinxstyleliteralemphasis{\sphinxupquote{bool}}}) \textendash{} Indicates if the provided angle is in degrees, if yes It will be converted to radians

\end{itemize}

\end{description}\end{quote}

Returns: The Rotational Matrix at the X axis by an \sphinxstyleemphasis{gamma} angle

\end{fulllineitems}

\index{rot\_y() (MatrixManipulation.Matrix method)}

\begin{fulllineitems}
\phantomsection\label{\detokenize{MatrixManipulation:MatrixManipulation.Matrix.rot_y}}\pysiglinewithargsret{\sphinxbfcode{\sphinxupquote{rot\_y}}}{\emph{beta=0}, \emph{degrees=True}}{}
Definition: Receives an theta angle and returns the rotation matrix for the given angle at the \sphinxstyleemphasis{Z} axis.
If the angle is given in radian degrees should be False.
\begin{quote}\begin{description}
\item[{Parameters}] \leavevmode\begin{itemize}
\item {} 
\sphinxstyleliteralstrong{\sphinxupquote{beta}} (\sphinxhref{https://docs.python.org/3/library/functions.html\#float}{\sphinxstyleliteralemphasis{\sphinxupquote{float}}}) \textendash{} Rotation Angle around the Z axis

\item {} 
\sphinxstyleliteralstrong{\sphinxupquote{degrees}} (\sphinxhref{https://docs.python.org/3/library/functions.html\#bool}{\sphinxstyleliteralemphasis{\sphinxupquote{bool}}}) \textendash{} Indicates if the provided angle is in degrees, if yes It will be converted to radians

\end{itemize}

\end{description}\end{quote}

Returns: The Rotational Matrix at the Z axis by an \sphinxstyleemphasis{beta} angle

\end{fulllineitems}

\index{rot\_z() (MatrixManipulation.Matrix method)}

\begin{fulllineitems}
\phantomsection\label{\detokenize{MatrixManipulation:MatrixManipulation.Matrix.rot_z}}\pysiglinewithargsret{\sphinxbfcode{\sphinxupquote{rot\_z}}}{\emph{alpha=0}, \emph{degrees=True}}{}
Definition: Receives an theta angle and returns the rotation matrix for the given angle at the \sphinxstyleemphasis{Z} axis.
If the angle is given in radian degrees should be False.
\begin{quote}\begin{description}
\item[{Parameters}] \leavevmode\begin{itemize}
\item {} 
\sphinxstyleliteralstrong{\sphinxupquote{alpha}} (\sphinxhref{https://docs.python.org/3/library/functions.html\#float}{\sphinxstyleliteralemphasis{\sphinxupquote{float}}}) \textendash{} Rotation Angle around the Z axis

\item {} 
\sphinxstyleliteralstrong{\sphinxupquote{degrees}} (\sphinxhref{https://docs.python.org/3/library/functions.html\#bool}{\sphinxstyleliteralemphasis{\sphinxupquote{bool}}}) \textendash{} Indicates if the provided angle is in degrees, if yes It will be converted to radians

\end{itemize}

\end{description}\end{quote}

Returns: The Rotational Matrix at the Z axis by an \sphinxstyleemphasis{alpha} angle

\end{fulllineitems}

\index{trans\_x() (MatrixManipulation.Matrix method)}

\begin{fulllineitems}
\phantomsection\label{\detokenize{MatrixManipulation:MatrixManipulation.Matrix.trans_x}}\pysiglinewithargsret{\sphinxbfcode{\sphinxupquote{trans\_x}}}{\emph{a=0}}{}
Definition: Translates the matrix a given amount \sphinxtitleref{a} on the \sphinxstyleemphasis{X} axis by Defining a 4x4 identity
matrix with \sphinxtitleref{a} as the (1,4) element.
\begin{quote}\begin{description}
\item[{Parameters}] \leavevmode
\sphinxstyleliteralstrong{\sphinxupquote{a}} (\sphinxhref{https://docs.python.org/3/library/functions.html\#float}{\sphinxstyleliteralemphasis{\sphinxupquote{float}}}) \textendash{} Distance translated on the X-axis

\end{description}\end{quote}

Returns: The Translation Matrix on the \sphinxstyleemphasis{X} axis by a distance \sphinxstyleemphasis{a}

\end{fulllineitems}

\index{trans\_y() (MatrixManipulation.Matrix method)}

\begin{fulllineitems}
\phantomsection\label{\detokenize{MatrixManipulation:MatrixManipulation.Matrix.trans_y}}\pysiglinewithargsret{\sphinxbfcode{\sphinxupquote{trans\_y}}}{\emph{b=0}}{}
Definition: Translate the matrix a given amount \sphinxtitleref{d} on the \sphinxstyleemphasis{Z} axis. by Defining a matrix T 4x4 identity
matrix with \sphinxstyleemphasis{b} (3,4) element position.
\begin{quote}\begin{description}
\item[{Parameters}] \leavevmode
\sphinxstyleliteralstrong{\sphinxupquote{b}} (\sphinxhref{https://docs.python.org/3/library/functions.html\#float}{\sphinxstyleliteralemphasis{\sphinxupquote{float}}}) \textendash{} Distance translated on the Z-axis

\end{description}\end{quote}

Returns: The Translation Matrix on the \sphinxstyleemphasis{Z} axis by a distance \sphinxstyleemphasis{b}

\end{fulllineitems}

\index{trans\_z() (MatrixManipulation.Matrix method)}

\begin{fulllineitems}
\phantomsection\label{\detokenize{MatrixManipulation:MatrixManipulation.Matrix.trans_z}}\pysiglinewithargsret{\sphinxbfcode{\sphinxupquote{trans\_z}}}{\emph{d=0}}{}
Definition: Translate the matrix a given amount \sphinxtitleref{d} on the \sphinxstyleemphasis{Z} axis. by Defining a matrix T 4x4 identity
matrix with \sphinxstyleemphasis{c} (3,4) element position.
\begin{quote}\begin{description}
\item[{Parameters}] \leavevmode
\sphinxstyleliteralstrong{\sphinxupquote{d}} (\sphinxhref{https://docs.python.org/3/library/functions.html\#float}{\sphinxstyleliteralemphasis{\sphinxupquote{float}}}) \textendash{} Distance translated on the Z-axis

\end{description}\end{quote}

Returns: The Translation Matrix on the \sphinxstyleemphasis{Z} axis by a distance \sphinxstyleemphasis{c}

\end{fulllineitems}


\end{fulllineitems}

\index{main() (in module MatrixManipulation)}

\begin{fulllineitems}
\phantomsection\label{\detokenize{MatrixManipulation:MatrixManipulation.main}}\pysiglinewithargsret{\sphinxcode{\sphinxupquote{MatrixManipulation.}}\sphinxbfcode{\sphinxupquote{main}}}{}{}
Example 3

\end{fulllineitems}



\chapter{MatrixManipulationSymbolic module}
\label{\detokenize{MatrixManipulationSymbolic:module-MatrixManipulationSymbolic}}\label{\detokenize{MatrixManipulationSymbolic:matrixmanipulationsymbolic-module}}\label{\detokenize{MatrixManipulationSymbolic::doc}}\index{MatrixManipulationSymbolic (module)}\index{MatrixSymbolic (class in MatrixManipulationSymbolic)}

\begin{fulllineitems}
\phantomsection\label{\detokenize{MatrixManipulationSymbolic:MatrixManipulationSymbolic.MatrixSymbolic}}\pysiglinewithargsret{\sphinxbfcode{\sphinxupquote{class }}\sphinxcode{\sphinxupquote{MatrixManipulationSymbolic.}}\sphinxbfcode{\sphinxupquote{MatrixSymbolic}}}{\emph{**kwargs}}{}
Bases: \sphinxhref{https://docs.python.org/3/library/functions.html\#object}{\sphinxcode{\sphinxupquote{object}}}

Definition: This class generates Homogeneous transform matrices, although it uses a symbolic approach
that can be used to multiply any matrix and obtain the translation or rotation.

It uses \sphinxtitleref{sympy} to generate the matrices:

sympy.Matrix: creates a sympy matrix object.

sympy.Symbol: creates a symbol, Symbols are identified by name and assumptions.
First, you need to create symbols using Symbol(“x”)
We are assuming here that the symbols are “Real” number.
All newly created symbols have assumptions set according to \sphinxtitleref{args}, for example:

\fvset{hllines={, ,}}%
\begin{sphinxVerbatim}[commandchars=\\\{\}]
\PYG{g+gp}{\PYGZgt{}\PYGZgt{}\PYGZgt{} }\PYG{n}{a} \PYG{o}{=} \PYG{n}{symbols}\PYG{p}{(}\PYG{l+s+s1}{\PYGZsq{}}\PYG{l+s+s1}{a}\PYG{l+s+s1}{\PYGZsq{}}\PYG{p}{,} \PYG{n}{integer}\PYG{o}{=}\PYG{k+kc}{True}\PYG{p}{)}
\PYG{g+gp}{\PYGZgt{}\PYGZgt{}\PYGZgt{} }\PYG{n}{a}\PYG{o}{.}\PYG{n}{is\PYGZus{}integer}
\PYG{g+go}{True}
\PYG{g+gp}{\PYGZgt{}\PYGZgt{}\PYGZgt{} }\PYG{n}{x}\PYG{p}{,} \PYG{n}{y}\PYG{p}{,} \PYG{n}{z} \PYG{o}{=} \PYG{n}{symbols}\PYG{p}{(}\PYG{l+s+s1}{\PYGZsq{}}\PYG{l+s+s1}{x,y,z}\PYG{l+s+s1}{\PYGZsq{}}\PYG{p}{,} \PYG{n}{real}\PYG{o}{=}\PYG{k+kc}{True}\PYG{p}{)}
\PYG{g+gp}{\PYGZgt{}\PYGZgt{}\PYGZgt{} }\PYG{n}{x}\PYG{o}{.}\PYG{n}{is\PYGZus{}real} \PYG{o+ow}{and} \PYG{n}{y}\PYG{o}{.}\PYG{n}{is\PYGZus{}real} \PYG{o+ow}{and} \PYG{n}{z}\PYG{o}{.}\PYG{n}{is\PYGZus{}real}
\PYG{g+go}{True}
\end{sphinxVerbatim}

sympy.cos and sympy.sin: cos and sin for sympy

sympy.simplify: SymPy has dozens of functions to perform various kinds of simplification.
simplify() attempts to apply all of these functions
in an intelligent way to arrive at the simplest form of an expression.

Returns: It returns Rotation and translation matrices.

Obs: {\color{red}\bfseries{}**}kwargs (keyword arguments) are used to facilitate the identification of the parameters, so initiate the
object
\index{rot\_x() (MatrixManipulationSymbolic.MatrixSymbolic method)}

\begin{fulllineitems}
\phantomsection\label{\detokenize{MatrixManipulationSymbolic:MatrixManipulationSymbolic.MatrixSymbolic.rot_x}}\pysiglinewithargsret{\sphinxbfcode{\sphinxupquote{rot\_x}}}{\emph{gamma='gamma\_i-1'}}{}
Definition: Receives an alpha angle and returns the rotation matrix for the given angle at the \sphinxstyleemphasis{X} axis.
If the angle is given in radian degrees should be False.
\begin{quote}\begin{description}
\item[{Parameters}] \leavevmode
\sphinxstyleliteralstrong{\sphinxupquote{gamma}} (\sphinxstyleliteralemphasis{\sphinxupquote{string}}) \textendash{} Rotation Angle around the X axis

\end{description}\end{quote}

Returns: The Rotational Matrix at the X axis by an \sphinxstyleemphasis{given} angle

\end{fulllineitems}

\index{rot\_y() (MatrixManipulationSymbolic.MatrixSymbolic method)}

\begin{fulllineitems}
\phantomsection\label{\detokenize{MatrixManipulationSymbolic:MatrixManipulationSymbolic.MatrixSymbolic.rot_y}}\pysiglinewithargsret{\sphinxbfcode{\sphinxupquote{rot\_y}}}{\emph{beta='beta\_i-1'}}{}
Definition: Receives an theta angle and returns the rotation matrix for the given angle at the \sphinxstyleemphasis{Z} axis.
If the angle is given in radian degrees should be False.
\begin{quote}\begin{description}
\item[{Parameters}] \leavevmode
\sphinxstyleliteralstrong{\sphinxupquote{beta}} (\sphinxstyleliteralemphasis{\sphinxupquote{string}}) \textendash{} Rotation Angle around the Y axis

\end{description}\end{quote}

Returns: The Rotational Matrix at the Y axis by an \sphinxstyleemphasis{given} angle

\end{fulllineitems}

\index{rot\_z() (MatrixManipulationSymbolic.MatrixSymbolic method)}

\begin{fulllineitems}
\phantomsection\label{\detokenize{MatrixManipulationSymbolic:MatrixManipulationSymbolic.MatrixSymbolic.rot_z}}\pysiglinewithargsret{\sphinxbfcode{\sphinxupquote{rot\_z}}}{\emph{alpha='alpha\_i-1'}}{}
Definition: Receives an theta angle and returns the rotation matrix for the given angle at the \sphinxstyleemphasis{Z} axis.
If the angle is given in radian degrees should be False.
\begin{quote}\begin{description}
\item[{Parameters}] \leavevmode
\sphinxstyleliteralstrong{\sphinxupquote{alpha}} (\sphinxstyleliteralemphasis{\sphinxupquote{string}}) \textendash{} Rotation Angle around the Z axis

\end{description}\end{quote}

Returns: The Rotational Matrix at the Z axis by an \sphinxstyleemphasis{given} angle

\end{fulllineitems}

\index{trans\_x() (MatrixManipulationSymbolic.MatrixSymbolic method)}

\begin{fulllineitems}
\phantomsection\label{\detokenize{MatrixManipulationSymbolic:MatrixManipulationSymbolic.MatrixSymbolic.trans_x}}\pysiglinewithargsret{\sphinxbfcode{\sphinxupquote{trans\_x}}}{\emph{a='a\_i-1'}}{}
Definition: Translates the matrix a given amount \sphinxtitleref{a} on the \sphinxstyleemphasis{X} axis by Defining a 4x4 identity
matrix with \sphinxtitleref{a} as the (1,4) element.
\begin{quote}\begin{description}
\item[{Parameters}] \leavevmode
\sphinxstyleliteralstrong{\sphinxupquote{a}} (\sphinxstyleliteralemphasis{\sphinxupquote{string}}) \textendash{} Distance translated on the X-axis

\end{description}\end{quote}

Returns: The Translation Matrix on the \sphinxstyleemphasis{X} axis by a given distance

\end{fulllineitems}

\index{trans\_y() (MatrixManipulationSymbolic.MatrixSymbolic method)}

\begin{fulllineitems}
\phantomsection\label{\detokenize{MatrixManipulationSymbolic:MatrixManipulationSymbolic.MatrixSymbolic.trans_y}}\pysiglinewithargsret{\sphinxbfcode{\sphinxupquote{trans\_y}}}{\emph{b='b\_i-1'}}{}
Definition: Translate the matrix a given amount \sphinxtitleref{d} on the \sphinxstyleemphasis{Z} axis. by Defining a matrix T 4x4 identity
matrix with \sphinxstyleemphasis{b} (3,4) element position.
\begin{quote}\begin{description}
\item[{Parameters}] \leavevmode
\sphinxstyleliteralstrong{\sphinxupquote{b}} (\sphinxstyleliteralemphasis{\sphinxupquote{string}}) \textendash{} Distance translated on the Z-axis

\end{description}\end{quote}

Returns: The Translation Matrix on the \sphinxstyleemphasis{Z} axis by a given distance

\end{fulllineitems}

\index{trans\_z() (MatrixManipulationSymbolic.MatrixSymbolic method)}

\begin{fulllineitems}
\phantomsection\label{\detokenize{MatrixManipulationSymbolic:MatrixManipulationSymbolic.MatrixSymbolic.trans_z}}\pysiglinewithargsret{\sphinxbfcode{\sphinxupquote{trans\_z}}}{\emph{d='d\_i-1'}}{}
Definition: Translate the matrix a given amount \sphinxtitleref{d} on the \sphinxstyleemphasis{Z} axis. by Defining a matrix T 4x4 identity
matrix with \sphinxstyleemphasis{c} (3,4) element position.
\begin{quote}\begin{description}
\item[{Parameters}] \leavevmode
\sphinxstyleliteralstrong{\sphinxupquote{d}} (\sphinxstyleliteralemphasis{\sphinxupquote{string}}) \textendash{} Distance translated on the Z-axis

\end{description}\end{quote}

Returns: The Translation Matrix on the \sphinxstyleemphasis{Z} axis by a given distance

\end{fulllineitems}


\end{fulllineitems}

\index{main() (in module MatrixManipulationSymbolic)}

\begin{fulllineitems}
\phantomsection\label{\detokenize{MatrixManipulationSymbolic:MatrixManipulationSymbolic.main}}\pysiglinewithargsret{\sphinxcode{\sphinxupquote{MatrixManipulationSymbolic.}}\sphinxbfcode{\sphinxupquote{main}}}{}{}
Example 6:

Calculates the Three-link manipulator kinematics.
At the end we can express a Transform from link 0 to link 3.

\end{fulllineitems}



\chapter{Example7 module}
\label{\detokenize{Example7:example7-module}}\label{\detokenize{Example7::doc}}
\fvset{hllines={, ,}}%
\begin{sphinxVerbatim}[commandchars=\\\{\},numbers=left,firstnumber=1,stepnumber=1]
\PYG{k+kn}{import} \PYG{n+nn}{sympy} \PYG{k}{as} \PYG{n+nn}{sympy}
\PYG{k+kn}{from} \PYG{n+nn}{src}\PYG{n+nn}{.}\PYG{n+nn}{MatrixManipulationSymbolic} \PYG{k+kn}{import} \PYG{n}{MatrixSymbolic}


\PYG{k}{def} \PYG{n+nf}{main}\PYG{p}{(}\PYG{p}{)}\PYG{p}{:}
    \PYG{l+s+sd}{\PYGZdq{}\PYGZdq{}\PYGZdq{}}
\PYG{l+s+sd}{    Example 7, First part homogeneous transform:}

\PYG{l+s+sd}{    Calculates the Three\PYGZhy{}link manipulator kinematics.}
\PYG{l+s+sd}{    At the end we can express a Transform from link 0 to link 4.}
\PYG{l+s+sd}{    \PYGZdq{}\PYGZdq{}\PYGZdq{}}
    \PYG{n+nb}{print}\PYG{p}{(}\PYG{l+s+s1}{\PYGZsq{}}\PYG{l+s+s1}{Example 7:}\PYG{l+s+s1}{\PYGZsq{}}\PYG{p}{)}

    \PYG{n}{a1} \PYG{o}{=} \PYG{n}{MatrixSymbolic}\PYG{p}{(}\PYG{p}{)}       \PYG{c+c1}{\PYGZsh{} Rx(a\PYGZus{}i\PYGZhy{}1)}
    \PYG{n}{a2} \PYG{o}{=} \PYG{n}{MatrixSymbolic}\PYG{p}{(}\PYG{p}{)}       \PYG{c+c1}{\PYGZsh{} Dx(a\PYGZus{}i\PYGZhy{}1)}
    \PYG{n}{a3} \PYG{o}{=} \PYG{n}{MatrixSymbolic}\PYG{p}{(}\PYG{p}{)}       \PYG{c+c1}{\PYGZsh{} Dz(d\PYGZus{}i)}
    \PYG{n}{a4} \PYG{o}{=} \PYG{n}{MatrixSymbolic}\PYG{p}{(}\PYG{p}{)}       \PYG{c+c1}{\PYGZsh{} Rz(theta\PYGZus{}i)}

    \PYG{n+nb}{print}\PYG{p}{(}\PYG{p}{)}
    \PYG{n+nb}{print}\PYG{p}{(}\PYG{l+s+s1}{\PYGZsq{}}\PYG{l+s+s1}{t\PYGZus{}0\PYGZus{}1:}\PYG{l+s+s1}{\PYGZsq{}}\PYG{p}{)}
    \PYG{n}{t\PYGZus{}0\PYGZus{}1} \PYG{o}{=} \PYG{p}{(}\PYG{n}{a1}\PYG{o}{.}\PYG{n}{rot\PYGZus{}x}\PYG{p}{(}\PYG{l+s+s1}{\PYGZsq{}}\PYG{l+s+s1}{0}\PYG{l+s+s1}{\PYGZsq{}}\PYG{p}{)}\PYG{p}{)} \PYG{o}{*} \PYG{p}{(}\PYG{n}{a2}\PYG{o}{.}\PYG{n}{trans\PYGZus{}x}\PYG{p}{(}\PYG{l+s+s1}{\PYGZsq{}}\PYG{l+s+s1}{0}\PYG{l+s+s1}{\PYGZsq{}}\PYG{p}{)}\PYG{p}{)} \PYG{o}{*} \PYG{p}{(}\PYG{n}{a3}\PYG{o}{.}\PYG{n}{trans\PYGZus{}z}\PYG{p}{(}\PYG{l+s+s1}{\PYGZsq{}}\PYG{l+s+s1}{0}\PYG{l+s+s1}{\PYGZsq{}}\PYG{p}{)}\PYG{p}{)} \PYG{o}{*} \PYG{p}{(}\PYG{n}{a4}\PYG{o}{.}\PYG{n}{rot\PYGZus{}z}\PYG{p}{(}\PYG{l+s+s1}{\PYGZsq{}}\PYG{l+s+s1}{theta\PYGZus{}1}\PYG{l+s+s1}{\PYGZsq{}}\PYG{p}{)}\PYG{p}{)}
    \PYG{n+nb}{print}\PYG{p}{(}\PYG{n}{sympy}\PYG{o}{.}\PYG{n}{pretty}\PYG{p}{(}\PYG{n}{t\PYGZus{}0\PYGZus{}1}\PYG{p}{)}\PYG{p}{)}

    \PYG{n+nb}{print}\PYG{p}{(}\PYG{l+s+s1}{\PYGZsq{}}\PYG{l+s+s1}{t\PYGZus{}1\PYGZus{}2:}\PYG{l+s+s1}{\PYGZsq{}}\PYG{p}{)}
    \PYG{n}{t\PYGZus{}1\PYGZus{}2} \PYG{o}{=} \PYG{p}{(}\PYG{n}{a1}\PYG{o}{.}\PYG{n}{rot\PYGZus{}x}\PYG{p}{(}\PYG{l+s+s1}{\PYGZsq{}}\PYG{l+s+s1}{90.0}\PYG{l+s+s1}{\PYGZsq{}}\PYG{p}{)}\PYG{p}{)} \PYG{o}{*} \PYG{p}{(}\PYG{n}{a2}\PYG{o}{.}\PYG{n}{trans\PYGZus{}x}\PYG{p}{(}\PYG{l+s+s1}{\PYGZsq{}}\PYG{l+s+s1}{0}\PYG{l+s+s1}{\PYGZsq{}}\PYG{p}{)}\PYG{p}{)} \PYG{o}{*} \PYG{p}{(}\PYG{n}{a3}\PYG{o}{.}\PYG{n}{trans\PYGZus{}z}\PYG{p}{(}\PYG{l+s+s1}{\PYGZsq{}}\PYG{l+s+s1}{0}\PYG{l+s+s1}{\PYGZsq{}}\PYG{p}{)}\PYG{p}{)} \PYG{o}{*} \PYG{p}{(}\PYG{n}{a4}\PYG{o}{.}\PYG{n}{rot\PYGZus{}z}\PYG{p}{(}\PYG{l+s+s1}{\PYGZsq{}}\PYG{l+s+s1}{theta\PYGZus{}2}\PYG{l+s+s1}{\PYGZsq{}}\PYG{p}{)}\PYG{p}{)}
    \PYG{n+nb}{print}\PYG{p}{(}\PYG{n}{sympy}\PYG{o}{.}\PYG{n}{pretty}\PYG{p}{(}\PYG{n}{t\PYGZus{}1\PYGZus{}2}\PYG{p}{)}\PYG{p}{)}

    \PYG{n+nb}{print}\PYG{p}{(}\PYG{p}{)}
    \PYG{n+nb}{print}\PYG{p}{(}\PYG{l+s+s1}{\PYGZsq{}}\PYG{l+s+s1}{t\PYGZus{}2\PYGZus{}3:}\PYG{l+s+s1}{\PYGZsq{}}\PYG{p}{)}
    \PYG{n}{t\PYGZus{}2\PYGZus{}3} \PYG{o}{=} \PYG{p}{(}\PYG{n}{a1}\PYG{o}{.}\PYG{n}{rot\PYGZus{}x}\PYG{p}{(}\PYG{l+s+s1}{\PYGZsq{}}\PYG{l+s+s1}{0}\PYG{l+s+s1}{\PYGZsq{}}\PYG{p}{)}\PYG{p}{)} \PYG{o}{*} \PYG{p}{(}\PYG{n}{a2}\PYG{o}{.}\PYG{n}{trans\PYGZus{}x}\PYG{p}{(}\PYG{l+s+s1}{\PYGZsq{}}\PYG{l+s+s1}{l2}\PYG{l+s+s1}{\PYGZsq{}}\PYG{p}{)}\PYG{p}{)} \PYG{o}{*} \PYG{p}{(}\PYG{n}{a3}\PYG{o}{.}\PYG{n}{trans\PYGZus{}z}\PYG{p}{(}\PYG{l+s+s1}{\PYGZsq{}}\PYG{l+s+s1}{0}\PYG{l+s+s1}{\PYGZsq{}}\PYG{p}{)}\PYG{p}{)} \PYG{o}{*} \PYG{p}{(}\PYG{n}{a4}\PYG{o}{.}\PYG{n}{rot\PYGZus{}z}\PYG{p}{(}\PYG{l+s+s1}{\PYGZsq{}}\PYG{l+s+s1}{theta\PYGZus{}3}\PYG{l+s+s1}{\PYGZsq{}}\PYG{p}{)}\PYG{p}{)}
    \PYG{n+nb}{print}\PYG{p}{(}\PYG{n}{sympy}\PYG{o}{.}\PYG{n}{pretty}\PYG{p}{(}\PYG{n}{t\PYGZus{}2\PYGZus{}3}\PYG{p}{)}\PYG{p}{)}

    \PYG{n+nb}{print}\PYG{p}{(}\PYG{p}{)}
    \PYG{n+nb}{print}\PYG{p}{(}\PYG{l+s+s1}{\PYGZsq{}}\PYG{l+s+s1}{t\PYGZus{}3\PYGZus{}4:}\PYG{l+s+s1}{\PYGZsq{}}\PYG{p}{)}
    \PYG{n}{t\PYGZus{}3\PYGZus{}4} \PYG{o}{=} \PYG{p}{(}\PYG{n}{a1}\PYG{o}{.}\PYG{n}{rot\PYGZus{}x}\PYG{p}{(}\PYG{l+s+s1}{\PYGZsq{}}\PYG{l+s+s1}{0}\PYG{l+s+s1}{\PYGZsq{}}\PYG{p}{)}\PYG{p}{)} \PYG{o}{*} \PYG{p}{(}\PYG{n}{a2}\PYG{o}{.}\PYG{n}{trans\PYGZus{}x}\PYG{p}{(}\PYG{l+s+s1}{\PYGZsq{}}\PYG{l+s+s1}{l3}\PYG{l+s+s1}{\PYGZsq{}}\PYG{p}{)}\PYG{p}{)} \PYG{o}{*} \PYG{p}{(}\PYG{n}{a3}\PYG{o}{.}\PYG{n}{trans\PYGZus{}z}\PYG{p}{(}\PYG{l+s+s1}{\PYGZsq{}}\PYG{l+s+s1}{0}\PYG{l+s+s1}{\PYGZsq{}}\PYG{p}{)}\PYG{p}{)} \PYG{o}{*} \PYG{p}{(}\PYG{n}{a4}\PYG{o}{.}\PYG{n}{rot\PYGZus{}z}\PYG{p}{(}\PYG{l+s+s1}{\PYGZsq{}}\PYG{l+s+s1}{0}\PYG{l+s+s1}{\PYGZsq{}}\PYG{p}{)}\PYG{p}{)}
    \PYG{n+nb}{print}\PYG{p}{(}\PYG{n}{sympy}\PYG{o}{.}\PYG{n}{pretty}\PYG{p}{(}\PYG{n}{t\PYGZus{}3\PYGZus{}4}\PYG{p}{)}\PYG{p}{)}

    \PYG{n}{t\PYGZus{}0\PYGZus{}4} \PYG{o}{=} \PYG{n}{t\PYGZus{}0\PYGZus{}1} \PYG{o}{*} \PYG{n}{t\PYGZus{}1\PYGZus{}2} \PYG{o}{*} \PYG{n}{t\PYGZus{}2\PYGZus{}3} \PYG{o}{*} \PYG{n}{t\PYGZus{}3\PYGZus{}4}

    \PYG{n+nb}{print}\PYG{p}{(}\PYG{p}{)}
    \PYG{n+nb}{print}\PYG{p}{(}\PYG{l+s+s1}{\PYGZsq{}}\PYG{l+s+s1}{t\PYGZus{}0\PYGZus{}4:}\PYG{l+s+s1}{\PYGZsq{}}\PYG{p}{)}
    \PYG{n+nb}{print}\PYG{p}{(}\PYG{n}{sympy}\PYG{o}{.}\PYG{n}{pretty}\PYG{p}{(}\PYG{n}{sympy}\PYG{o}{.}\PYG{n}{simplify}\PYG{p}{(}\PYG{n}{t\PYGZus{}0\PYGZus{}4}\PYG{p}{)}\PYG{p}{)}\PYG{p}{)}

    \PYG{n}{t\PYGZus{}0\PYGZus{}4f} \PYG{o}{=} \PYG{n}{t\PYGZus{}0\PYGZus{}4}\PYG{o}{.}\PYG{n}{evalf}\PYG{p}{(}\PYG{p}{)}

    \PYG{n+nb}{print}\PYG{p}{(}\PYG{p}{)}
    \PYG{n+nb}{print}\PYG{p}{(}\PYG{l+s+s1}{\PYGZsq{}}\PYG{l+s+s1}{t\PYGZus{}0\PYGZus{}4f:}\PYG{l+s+s1}{\PYGZsq{}}\PYG{p}{)}
    \PYG{n+nb}{print}\PYG{p}{(}\PYG{n}{sympy}\PYG{o}{.}\PYG{n}{pretty}\PYG{p}{(}\PYG{n}{sympy}\PYG{o}{.}\PYG{n}{simplify}\PYG{p}{(}\PYG{n}{t\PYGZus{}0\PYGZus{}4f}\PYG{p}{)}\PYG{p}{)}\PYG{p}{)}

    \PYG{k}{return}


\PYG{k}{if} \PYG{n+nv+vm}{\PYGZus{}\PYGZus{}name\PYGZus{}\PYGZus{}} \PYG{o}{==} \PYG{l+s+s1}{\PYGZsq{}}\PYG{l+s+s1}{\PYGZus{}\PYGZus{}main\PYGZus{}\PYGZus{}}\PYG{l+s+s1}{\PYGZsq{}}\PYG{p}{:}
    \PYG{n}{main}\PYG{p}{(}\PYG{p}{)}
\end{sphinxVerbatim}


\chapter{Indices and tables}
\label{\detokenize{index:indices-and-tables}}
At the website you can navigate through the menus below:
\begin{itemize}
\item {} 
\DUrole{xref,std,std-ref}{genindex}

\item {} 
\DUrole{xref,std,std-ref}{modindex}

\item {} 
\DUrole{xref,std,std-ref}{search}

\end{itemize}


\section{Running the documentation with Sphinx}
\label{\detokenize{index:running-the-documentation-with-sphinx}}
To run the documentation for this project run the following commands, at the project folder:
\begin{quote}

Install Spinxs:

\sphinxstylestrong{python -m pip install sphinx}

Install the “Read the Docs” theme:

\sphinxstylestrong{pip install sphinx-rtd-theme}

\sphinxstylestrong{make clean}

\sphinxstylestrong{make html}
\end{quote}


\section{GitHub Repository}
\label{\detokenize{index:github-repository}}
Find all the files at the GitHub repository \sphinxhref{https://github.com/ThiagoSoutoGit/OpenCVMatrices}{here}.


\renewcommand{\indexname}{Python Module Index}
\begin{sphinxtheindex}
\def\bigletter#1{{\Large\sffamily#1}\nopagebreak\vspace{1mm}}
\bigletter{m}
\item {\sphinxstyleindexentry{MatrixManipulation}}\sphinxstyleindexpageref{MatrixManipulation:\detokenize{module-MatrixManipulation}}
\item {\sphinxstyleindexentry{MatrixManipulationSymbolic}}\sphinxstyleindexpageref{MatrixManipulationSymbolic:\detokenize{module-MatrixManipulationSymbolic}}
\end{sphinxtheindex}

\renewcommand{\indexname}{Index}
\printindex
\end{document}